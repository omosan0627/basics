\documentclass[a4paper,10pt]{article}
\usepackage{amsmath,amssymb}
\usepackage[pdftex]{graphicx}
\usepackage{nccmath}
\usepackage{here}

\newtheorem{definition}{Definition}[section]
\newtheorem{thm}{\bfseries Theorem}[section]
\newtheorem{lem}[thm]{\bfseries Lemma}        %% lemmas, props, cor, etc
\newtheorem{remark}[thm]{\bfseries Remark}    %%   are numbered consecutively
\newtheorem{iden}{\bfseries Identity}         %%   with the theorems.
\newtheorem{sublem}{\bfseries Sub-lemma}[thm] %% 
\newtheorem{prop}[thm]{\bfseries Proposition} %%
\newtheorem{cor}[thm]{\bfseries Corollary}     
\newtheorem{defn}[thm]{\bfseries Definition}
\newtheorem{cl}[thm]{\bfseries Claim}
\newtheorem{axiom}[thm]{\bfseries Axiom}
\newtheorem{conj}[thm]{\bfseries Conjecture}
\newtheorem{fact}[thm]{\bfseries Fact}
\newtheorem{hypo}[thm]{\bfseries Hypothesis}
\newtheorem{assum}[thm]{\bfseries Assumption}
\newtheorem{crit}[thm]{\bfseries Criterion}
\newtheorem{exmp}[thm]{\bfseries Example}
\newtheorem{prob}[thm]{\bfseries Problem}
\newtheorem{prin}[thm]{\bfseries Principle}

\title{TaPL}
\author{omosan0627}
\begin{document}
\maketitle

\section*{3型無し算術式}
抽象構文? 帰納的定義・証明? 評価? 実行時エラーのモデル化?\\
Chapter5: 型無しラムダ、名前束縛、代入 Chapter8: 型システム、静的型付け\\
Chapter9: 静的型付けラムダ\\
\subsection*{3.1導入}
文法: 本書ではBNF\\
構文: 項、値などの組かな.\\
項: 計算の構文的表現(つまりメタ変数tに代入することができる表現)\\
式: あらゆる種類の構文的表現(項式、条件式など)??\\
メタ変数: メタ言語の変数. (対象言語の変数ではなく)\\
抽象構文:?\\
値: 項の部分集合で、評価の結果\\
\subsection*{3.2構文}
帰納的な項の定義: 推論規則を満たす最小の集合
具体的な項の定義: 前提を持つ規則を1回適用した項を集める。それを有限回繰り返して
得られる集合\\
完全帰納法は全て帰納ステップになっているとみなせる。\\
\subsection*{3.3項に関する帰納法}
構造的帰納法:
「各項$s$に対して、$s$の任意の直接の部分項$r$に対して$P(r)$がなりたつとき、
$P(s)$が証明できる」ならば、全ての$s$に対して$P(s)$が成立する
これは具体的な項の定義から証明できる. これは一般的な帰納法になっている.
\subsection*{3.4意味論のスタイル}
表示的意味論(モデル理論っぽ)、公理的意味論(ホーア論理とか?)もあるが、
操作的意味論をこの本では扱う
\subsection*{3.5評価}
評価関係: 関係は(項,項)の集合であることに注意.\\
インスタンス: 規則の結論や前提のメタ変数それぞれに対し、一貫して
同じ項による置き換えを行ったものである.\\
規則がある関係によって満たされる: 規則の任意のインスタンスについて,
結論がその関係に属するか, または前提の内の一つが属さないことである.\\
1ステップ評価関係:規則を満たす最小の二項関係. これは項の定義同様具体的に構成できるし、
構造的帰納法も使える. 導出に関する帰納法と言う. \\
1ステップ評価の決定性: Coqでの証明ができません.
項tが正規形:t $\rightarrow$ t'となるt'が存在しないとき.全ての値は正規形である. 
正規形は値とは限らず、そうでないとき行き詰まりという。状態実行時エラーの解析に使われるかも. \\
多ステップ評価関係$\rightarrow^*$: 1ステップ評価の反射的推移的閉包. つまり有限回の
1ステップ評価で到達できる項の関係. これも推論規則から定義できます. \\
停止尺度: 評価の停止性の証明で使われる関数のこと. 項について単調減少.
構文要素ってなんだ?
\section*{型無しラムダ計算}
\subsection*{基礎}

\end{document}