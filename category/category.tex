\documentclass[a4paper,10pt]{article}
\usepackage{amsmath,amssymb}
\usepackage{amsthm}
\usepackage[pdftex]{graphicx}
\usepackage{here}
\usepackage{url}
\usepackage{enumerate}
\usepackage[all]{xy}

\theoremstyle{definition}
\newtheorem{thm}{\bfseries Theorem}[section]
\newtheorem{definition}[thm]{\bfseries Definition}
\newtheorem{lem}[thm]{\bfseries Lemma}        %% lemmas, props, cor, etc
\newtheorem{remark}[thm]{\bfseries Remark}    %%   are numbered consecutively
\newtheorem{iden}{\bfseries Identity}         %%   with the theorems.
\newtheorem{sublem}{\bfseries Sub-lemma}[thm] %% 
\newtheorem{prop}[thm]{\bfseries Proposition} %%
\newtheorem{cor}[thm]{\bfseries Corollary}     
\newtheorem{defn}[thm]{\bfseries Definition}
\newtheorem{cl}[thm]{\bfseries Claim}
\newtheorem{axiom}[thm]{\bfseries Axiom}
\newtheorem{conj}[thm]{\bfseries Conjecture}
\newtheorem{fact}[thm]{\bfseries Fact}
\newtheorem{hypo}[thm]{\bfseries Hypothesis}
\newtheorem{assum}[thm]{\bfseries Assumption}
\newtheorem{crit}[thm]{\bfseries Criterion}
\newtheorem{exmp}[thm]{\bfseries Example}
\newtheorem{prob}[thm]{\bfseries Problem}
\newtheorem{prin}[thm]{\bfseries Principle}

\title{圏論}
\author{omosan0627}
\begin{document}
\maketitle

とくに断らない限り、圏はlocally smallとする. (小圏とは違うよ)
\section{圏論入門}
\subsection{圏論とは何か}
\url{http://alg-d.com/math/kan_extension/intro.pdf}
\begin{definition}
    圏$C$とは二つの集まり$\mathrm{Ob}(C)$, $\mathrm{Mor}(C)$の組であって, 以下の条件を満たすものをいう. 
    なお元$a \in \mathrm{Ob}(C)$を対象, $f \in \mathrm{Mor}(C)$を射と呼ぶ.
    \begin{enumerate}[(1)]
        \item 各$f \in \mathrm{Mor}(C)$に対して, ドメインと呼ばれる対象$\mathrm{dom}(f)\in \mathrm{Ob}(C)$
        とコドメインと呼ばれる対象$\mathrm{cod}(f) \in \mathrm{Ob}(C)$が定められている.
        $\mathrm{dom}(f) = a$, $\mathrm{cod}(f) = b$であることを$f:a \rightarrow b$や$a \xrightarrow{f} b$と書いて表す.
        また対象$a,b  \in \mathrm{Ob}(C)$に対して$\mathrm{Hom}_C(a,b):=\{f \in \mathrm{Mor}(C): a \xrightarrow{f} b\}$と書く.
        \item 2つの射$f,g \in \mathrm{Mor}(C)$について$\mathrm{cod}(f) = \mathrm{dom}(g)$であるとき, 
        $f$と$g$の合成射とよばれる射$g \circ f$が定められていて, $\mathrm{dom}(g \circ f)=\mathrm{dom}(f), \mathrm{cod}(g \circ f)=\mathrm{cod}(g)$
        を満たす。
        \item 射の合成は結合則を満たす. ($h \circ (g \circ f) = (h \circ g) \circ f$)
        \item 各$a \in \mathrm{Ob}(C)$に対して, 恒等射と呼ばれる射$\mathrm{id}_{a}:a \rightarrow a$が存在し, 射の合成
        に関する単位元となる. すなわち$f:a \rightarrow b$に対して, $f \circ \mathrm{id}_a = f, \mathrm{id}_b \circ f = f$である.
    \end{enumerate}
\end{definition}
\begin{exmp}
    $\mathbf{Set}, \mathbf{Grp}, \mathbf{Top}$
\end{exmp}
\begin{definition}
    $C,D$を圏とする. $C$から$D$への関手$F:C \rightarrow D$とは$a \in \mathrm{Ob}(C)$に$F(a) 
    \in \mathrm{Ob}(D)$を, 
    $f \in \mathrm{Mor}(C)$に$F(f) \in \mathrm{Mor}(D)$を対応させる関数であって, 以下を満たすものである.
    \begin{enumerate}[(1)]
        \item $f: a \rightarrow b$のとき$F(f): F(a) \rightarrow F(b)$である.
        \item $\mathrm{cod}(f) = \mathrm{dom}(g)$のとき,$F(g \circ f)= F(g) \circ F(f)$である.
        \item $a \in C$に対して$F(\mathrm{id}_a) = \mathrm{id}_{F(a)}$である.
    \end{enumerate}
\end{definition}
\begin{definition}
    $C$を圏, $a,b \in C$を対象とする.
    \begin{enumerate}[(1)]
        \item $C$の射$f:a \rightarrow b$が同型射\\
        $\Longleftrightarrow$ある射$g:b \rightarrow a$が存在して, $g \circ f = \mathrm{id}_a, f \circ g = \mathrm{id}_b$となる
        \item $a$と$b$が同型($a \cong b$で表す) $\Longleftrightarrow$ ある同型射$f:a \rightarrow b$が存在する.
    \end{enumerate}
\end{definition}
\begin{thm}
    $f$が同型射ならば$F(f)$も同型射
\end{thm}

\begin{definition}
    圏$C$と圏$D$が同型($C \cong D$と書く)とは, ある関手$F: C \rightarrow D,
    G: D \rightarrow C$が存在して$GF= \mathrm{\mathrm{id}}_C, FG=\mathrm{id}_D$.
\end{definition}



\subsection{自然変換・圏同値}
\url{http://alg-d.com/math/kan_extension/equivalence.pdf}
\begin{definition}
    $C,D$を圏, $F,G:C \rightarrow D$を関手とする. $F$から$G$への自然変換とは, $D$の射の族
    $\theta=\{\theta_a: Fa \rightarrow Fb\}_{a\in \mathrm{Ob}(C)}$であって, $C$の射$f:a \rightarrow b$
    に対して$Gf \circ \theta_a = \theta_b Ff$を満たすものをいう. (またこのとき$\theta_a$は$a$について
    自然という言い方をする.) 絵で書けば以下のようになる. 
    \begin{align*}
\xymatrix{
a\ar@{|->}@/^18pt/@<0.5ex>[rr]^-{F} \ar[d]^-{f} 
\ar@{|->}@/^36pt/@<2.0ex>[rrr]^-{G}& & 
Fa\ar[r]^-{\theta_a} \ar[d]^{Ff}& Ga \ar[d]^-{Gf} \\
b& 
& Fb\ar[r]^-{\theta_b}& Gb
}
    \end{align*}
$\theta$が$F$から$G$への自然変換であることを記号で$\theta: F \Rightarrow G$と表す.
また$\theta_a$を$\theta$の$a$成分と呼ぶ.
\end{definition}
\begin{definition}
    各$\theta_a$が同型射となる自然変換$\theta$を自然同型という. また自然同型
    $F \Rightarrow G$が存在するとき, $F$と$G$は自然同型であるといい,
    記号で$F \cong G$と表す.
\end{definition}
\begin{exmp}
    有限次元線形空間$V$と$V^{**}$についての自然変換$\theta: \mathrm{id}_c 
    \Rightarrow F \circ F^{\mathrm{op}}, \theta_V(x)(\rho) \mapsto \rho(x)$.
    線形代数の世界p135も参照. $V^*$の場合と違って, 基底を出さなくても自然変換が
    作れるところがポイント.
\end{exmp}
\begin{definition}
    圏$C, D$が圏同値($C \simeq D$と書く)\\
    $\Longleftrightarrow$ 関手$F: C \rightarrow D, G: D \rightarrow C$
    と自然変換$GF \cong \mathrm{id}_C, FG \cong \mathrm{id}_D$が存在する. 
\end{definition}



\end{document}