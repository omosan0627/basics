\documentclass[a4paper,10pt]{article}
\usepackage{amsmath,amssymb}
\usepackage{amsthm}
\usepackage[pdftex]{graphicx}
\usepackage{here}
\usepackage{bm}
\usepackage{url}
\usepackage{enumerate}
\usepackage[all]{xy}

\theoremstyle{definition}
\newtheorem{thm}{\bfseries Theorem}[section]
\newtheorem{definition}[thm]{\bfseries Definition}
\newtheorem{lem}[thm]{\bfseries Lemma}        %% lemmas, props, cor, etc
\newtheorem{remark}[thm]{\bfseries Remark}    %%   are numbered consecutively
\newtheorem{iden}{\bfseries Identity}         %%   with the theorems.
\newtheorem{sublem}{\bfseries Sub-lemma}[thm] %% 
\newtheorem{prop}[thm]{\bfseries Proposition} %%
\newtheorem{cor}[thm]{\bfseries Corollary}     
\newtheorem{defn}[thm]{\bfseries Definition}
\newtheorem{cl}[thm]{\bfseries Claim}
\newtheorem{axiom}[thm]{\bfseries Axiom}
\newtheorem{conj}[thm]{\bfseries Conjecture}
\newtheorem{fact}[thm]{\bfseries Fact}
\newtheorem{hypo}[thm]{\bfseries Hypothesis}
\newtheorem{assum}[thm]{\bfseries Assumption}
\newtheorem{crit}[thm]{\bfseries Criterion}
\newtheorem{exmp}[thm]{\bfseries Example}
\newtheorem{prob}[thm]{\bfseries Problem}
\newtheorem{prin}[thm]{\bfseries Principle}

\title{圏論}
\author{omosan0627}
\begin{document}
\maketitle

とくに断らない限り、圏はlocally smallとする. (小圏とは違うよ)
\section{圏論入門}
\subsection{圏論とは何か}
\url{http://alg-d.com/math/kan_extension/intro.pdf}
\begin{definition}
    圏$C$とは二つの集まり$\mathrm{Ob}(C)$, $\mathrm{Mor}(C)$の組であって, 以下の条件を満たすものをいう. 
    なお元$a \in \mathrm{Ob}(C)$を対象, $f \in \mathrm{Mor}(C)$を射と呼ぶ.
    \begin{enumerate}[(1)]
        \item 各$f \in \mathrm{Mor}(C)$に対して, ドメインと呼ばれる対象$\mathrm{dom}(f)\in \mathrm{Ob}(C)$
        とコドメインと呼ばれる対象$\mathrm{cod}(f) \in \mathrm{Ob}(C)$が定められている.
        $\mathrm{dom}(f) = a$, $\mathrm{cod}(f) = b$であることを$f:a \rightarrow b$や$a \xrightarrow{f} b$と書いて表す.
        また対象$a,b  \in \mathrm{Ob}(C)$に対して$\mathrm{Hom}_C(a,b):=\{f \in \mathrm{Mor}(C): a \xrightarrow{f} b\}$と書く.
        \item 2つの射$f,g \in \mathrm{Mor}(C)$について$\mathrm{cod}(f) = \mathrm{dom}(g)$であるとき, 
        $f$と$g$の合成射とよばれる射$g \circ f \in \mathrm{Mor}(C)$が定められていて, $\mathrm{dom}(g \circ f)=\mathrm{dom}(f), \mathrm{cod}(g \circ f)=\mathrm{cod}(g)$
        を満たす。
        \item 射の合成は結合則を満たす. ($h \circ (g \circ f) = (h \circ g) \circ f$)
        \item 各$a \in \mathrm{Ob}(C)$に対して, 恒等射と呼ばれる射$\mathrm{id}_{a}:a \rightarrow a$が存在し, 射の合成
        に関する単位元となる. すなわち$f:a \rightarrow b$に対して, $f \circ \mathrm{id}_a = f, \mathrm{id}_b \circ f = f$である.
    \end{enumerate}
\end{definition}
\begin{remark}
    $C = (\textrm{Ob}(C), \textrm{Mor}(C), \textrm{cod}, \textrm{dom}, \textrm{id}, \circ)$と書き表すことも。
    \begin{itemize}
     \item $\textrm{Ob}(C), \textrm{Mor}(C)$が集まり
     \item $\textrm{cod}, \textrm{cod}$が$\text{Mor}(C) \rightarrow \text{Ob}(C)$の関数
     \item $\textrm{id}$が$\textrm{Ob}(C) \rightarrow \textrm{Mor}(C)$の関数
     \item $\circ$が$\textrm{Mor}(C) \times \textrm{Mor}(C) \rightarrow \textrm{Mor}(C)$の関数
    \end{itemize}
\end{remark}
\begin{exmp}
    $\mathbf{Set}, \mathbf{Grp}, \mathbf{Top}$
\end{exmp}
\begin{definition}
    $C,D$を圏とする. $C$から$D$への関手$F:C \rightarrow D$とは$a \in \mathrm{Ob}(C)$に$F(a) 
    \in \mathrm{Ob}(D)$を, 
    $f \in \mathrm{Mor}(C)$に$F(f) \in \mathrm{Mor}(D)$を対応させる関数であって, 以下を満たすものである.
    \begin{enumerate}[(1)]
        \item $f: a \rightarrow b$のとき$F(f): F(a) \rightarrow F(b)$である.
        \item $\mathrm{cod}(f) = \mathrm{dom}(g)$のとき,$F(g \circ f)= F(g) \circ F(f)$である.
        \item $a \in C$に対して$F(\mathrm{id}_a) = \mathrm{id}_{F(a)}$である.
    \end{enumerate}
\end{definition}
\begin{definition}
    $C$を圏, $a,b \in C$を対象とする.
    \begin{enumerate}[(1)]
        \item $C$の射$f:a \rightarrow b$が同型射\\
        $\Longleftrightarrow$ある射$g:b \rightarrow a$が存在して, $g \circ f = \mathrm{id}_a, f \circ g = \mathrm{id}_b$となる
        \item $a$と$b$が同型($a \cong b$で表す) $\Longleftrightarrow$ ある同型射$f:a \rightarrow b$が存在する.
    \end{enumerate}
\end{definition}
\begin{thm}
    $f$が同型射ならば$F(f)$も同型射
\end{thm}

\begin{definition}
    圏$C$と圏$D$が同型($C \cong D$と書く)とは, ある関手$F: C \rightarrow D,
    G: D \rightarrow C$が存在して$GF= \mathrm{\mathrm{id}}_C, FG=\mathrm{id}_D$.
\end{definition}
\begin{definition}
    $C$を圏とする. このとき$C^{\mathrm{op}}$を以下のように定める.
    \begin{itemize}
        \item 対象$a \in C$に対して新しい対象$a^{\mathrm{op}}$を用意し, $\mathrm{Ob}(C^{\mathrm{op}})
        := \{a^{\mathrm{op}}:a \in \mathrm{Ob}(C)\}$と定める. 
        \item 射$f \in C$に対して新しい射$f^{\mathrm{op}}$を用意し, $\mathrm{Mor}(C^{\mathrm{op}})
        := \{f^{\mathrm{op}}:f \in \mathrm{Ob}(C)\}$と定める. 
        \item $\mathrm{dom}(f^{\mathrm{op}}):=\mathrm{cod}(f)^{\mathrm{op}}, 
        \mathrm{cod}(f^{\mathrm{op}}):=\mathrm{dom}(f)^{\mathrm{op}}$と定める. 即ち$f:a \rightarrow b$
        のとき$f^{\mathrm{op}}: b^{\mathrm{op}} \rightarrow a^{\mathrm{op}}$である.
        \item $f^{\mathrm{op}}:a^{\mathrm{op}} \rightarrow b^{\mathrm{op}}, g^{\mathrm{op}}
        : b^{\mathrm{op}} \rightarrow c^{\mathrm{op}}$に対して射の合成
        $g^{\mathrm{op}}\circ f^{\mathrm{op}}: a^{\mathrm{op}} \rightarrow c^{\mathrm{op}}$
        を $g^{\mathrm{op}}\circ f^{\mathrm{op}}=(f \circ g)^{\mathrm{op}}$と定める.
        \item $\mathrm{id}_{a^{\mathrm{op}}}:= \mathrm{id}_a^{\mathrm{op}}$とする.
    \end{itemize}
    これを圏$C^{\mathrm{op}}$の反対圏と呼ぶ.
\end{definition}
\begin{definition}
    圏$C, D$の直積$C \times D$を以下のように定義する.
    \begin{itemize}
        \item 対象は「$C$の対象と$D$の対象の組」である.
        \item $\langle c,d \rangle$から$\langle c',d' \rangle$への射は成分ごとの射の組
        $\langle f:c \rightarrow c', g: d \rightarrow d' \rangle$である. つまり
        $\textrm{Hom}_{C \times D}(\langle c,d \rangle, \langle c',d' \rangle):=
        \textrm{Hom}_C(c,c') \times \textrm{Hom}_D(d,d')$となる.
        \item 射の合成は成分ごとに行う. 即ち$\langle g, g' \rangle \circ \langle f,f' \rangle
        :=\langle g \circ f, g' \circ f'\rangle$となる.
        \item $\langle c,d \rangle$の恒等射は$\textrm{id}_{\langle c,d \rangle}:=
        \langle \textrm{id}_{c}, \textrm{id}_d \rangle$である.
    \end{itemize}
\end{definition}



\subsection{自然変換・圏同値}
\url{http://alg-d.com/math/kan_extension/equivalence.pdf}
\begin{definition}
    $C,D$を圏, $F,G:C \rightarrow D$を関手とする. $F$から$G$への自然変換とは, $D$の射の族
    $\theta=\{\theta_a: Fa \rightarrow Fb\}_{a\in \mathrm{Ob}(C)}$であって, 
    $\forall (a \xrightarrow{f} b) \in \mathrm{Mor}(C).\  \theta_b \circ Ff = Gf \circ \theta_a$を満たすものをいう. 
    (またこのとき$\theta_a$は$a$について自然という言い方をする.) 絵で書けば以下のようになる. 
    \begin{align*}
\xymatrix{
a\ar@{|->}@/^18pt/@<0.5ex>[rr]^-{F} \ar[d]^-{f} 
\ar@{|->}@/^36pt/@<2.0ex>[rrr]^-{G}& & 
Fa\ar[r]^-{\theta_a} \ar[d]^{Ff}& Ga \ar[d]^-{Gf} \\
b& 
& Fb\ar[r]^-{\theta_b}& Gb
}
    \end{align*}
$\theta$が$F$から$G$への自然変換であることを記号で$\theta: F \Rightarrow G$と表す.
また$\theta_a$を$\theta$の$a$成分と呼ぶ.
\end{definition}
\begin{remark}
    任意の$f$について$Ff$から$Gf$への変換則が成り立っていると思うと分かりやすい?
    また射、関手についてだが、基本的には$\rightarrow$しか記号として使わなくて、
    こういう圏の内部に言及するときだけ$\mapsto$を使うイメージ.
\end{remark}
\begin{definition}
    各$\theta_a$が同型射となる自然変換$\theta$を自然同型という. また自然同型
    $F \Rightarrow G$が存在するとき, $F$と$G$は自然同型であるといい,
    記号で$F \cong G$と表す.
\end{definition}
\begin{exmp}
    有限次元線形空間$V$と$V^{**}$についての自然変換$\theta: \mathrm{id}_c 
    \Rightarrow F \circ F^{\mathrm{op}}, \theta_V(x)(\rho) \mapsto \rho(x)$.
    線形代数の世界p135も参照. $V^*$の場合と違って, 基底を出さなくても自然変換が
    作れるところがポイント.
\end{exmp}
\begin{definition}
    圏$C, D$が圏同値($\bm{C \simeq D}$と書く)\\
    $\Longleftrightarrow$ 関手$F: C \rightarrow D, G: D \rightarrow C$
    と自然変換$GF \cong \mathrm{id}_C, FG \cong \mathrm{id}_D$が存在する. 
\end{definition}
\begin{definition}
    $C, D$を圏, $F:C \rightarrow D$を関手とする. 
    \begin{enumerate}[(1)]
        \item $F$が忠実$\Longleftrightarrow$ 
        $\forall a,b \in \mathrm{Ob}(C).\  
        F: \mathrm{Hom}_C(a,b) \rightarrow \mathrm{Hom}_C(Fa,Fb)$が単射.
        \item $F$が充満$\Longleftrightarrow$ 
        $\forall a,b \in \mathrm{Ob}(C).\  
        F: \mathrm{Hom}_C(a,b) \rightarrow \mathrm{Hom}_C(Fa,Fb)$が全射.
        \item $F$がconservative$\Longleftrightarrow$ 
        $\forall f \in \mathrm{Mor}(C).\  
        Ff \text{が同型ならば} f \text{も同型である}$.
        \item $F$が本質的単射$\Longleftrightarrow$ 
        $\forall a, b \in \mathrm{Ob}(C).\  
        Fa \cong Fb \text{ならば} a \cong b$ \\
        ($\Longleftrightarrow Fa\text{と}Fb\text{に同型射が存在するならば, }
        a\text{と}b\text{にも同型射が存在する.}$)
        \item $F$が本質的全射$\Longleftrightarrow$ 
        $\forall d \in \mathrm{Ob}(D).\  
        \exists c \in \mathrm{Ob}(C).\ Fc \cong d$
    \end{enumerate}
\end{definition}
\begin{prop}
        忠実充満$\Longrightarrow$ conservative, 
        忠実$\land$ conservative $\Longrightarrow$ 本質的単射
\end{prop}
\begin{thm}
    \label{faithful}
    $F$が圏同値を与える $\Longleftrightarrow$
    $F$が忠実充満な本質的全射
\end{thm}
\begin{proof}
    $F$が圏同値を与えるという条件は, 
    $G: D \rightarrow C$と自然同型$\theta: GF \Rightarrow \mathrm{id}_C, \epsilon:
    \Rightarrow \mathrm{id}_D$を使って, 以下で表される. 
\begin{align*}
    \forall& (c \xrightarrow{f} c') \in \mathrm{Mor}(C), 
    (d \xrightarrow{g} d') \in \mathrm{Mor}(D).\\
&\xymatrix{
GFc \ar[r]^-{\theta_c} \ar[d]^-{GFf} & c \ar[d]^-{f}& 
FGd\ar[r]^-{\epsilon_d} \ar[d]^{FGg} & d \ar[d]^-{g} \\
GFc' \ar[r]^-{\theta_{c'}}& c' 
& FGd\ar[r]^-{\epsilon_{d'}}& d'
}
    \end{align*}
($\Longrightarrow$)$\epsilon$から本質的全射, $\theta$から忠実充満が示せる.
また主に$\theta_c$が同型なので逆向きの$\theta_c^{-1}$が存在することを使う.\\
($\Longleftarrow$)本質的全射$Fc \rightarrow d$から$\epsilon$と$G$を作る. 
$Gg$の定義がすこしトリッキーだが忠実充満の定義に帰れば自然. 
最後に$\theta$が自然同型であることを言えばいいが, $Fc \xrightarrow{Ff} Fc'$について$\epsilon$の自然変換の図式
を利用することで示せる. 

\end{proof}
\begin{thm}
    $F$が同型 $\Longleftrightarrow$
    $F$が忠実充満で, 対象について全単射
\end{thm}
\begin{proof}
($\Longleftarrow$) Theorem \ref{faithful}と似ているが、ここでは本質的全射ではなく全単射.
\end{proof}
\begin{definition}
    部分圏$C \subseteq D$が充満部分圏であるとは、任意の$a, b\in C$に対して
    $\textrm{Hom}_{C}(a,b) = \textrm{Hom}_{D}(a,b)$となることをいう.
\end{definition}
\begin{definition}
    圏$C$が骨格的 $\Longleftrightarrow$ $a \cong b$ならば$a=b$である
\end{definition}
\begin{definition}
    圏$C$の骨格とは, 骨格的な充満部分圏$S \subseteq C$であって条件
    \begin{center}
        任意の$c$に対して, ある$s\in S$が存在して$c \cong s$となる
    \end{center}
    を満たすものをいう.
\end{definition}
\begin{thm}
    任意の圏は骨格を持つ. また骨格は圏同型を除いて一意である.
\end{thm}
\begin{proof}
    骨格を持つことを示す際に選択公理が必要. 一意性は$F$を同型射を利用して作って
    Theorem \ref{faithful}を適用.
\end{proof}
\begin{thm}
    $C, D$を圏, $S \subseteq C, T \subseteq D$を骨格とする. このとき
    \begin{center}
        $C$と$D$が圏同値 $\Longleftrightarrow$ $S$と$T$が圏同型
    \end{center}
\end{thm}
\begin{proof}
    ($\Longrightarrow$) $F$を$S$に制限した関手$F|_S$が圏同型であることを使うとできる. 本質的単射
    を使う.\\
    ($\Longleftarrow$) 包含関手が圏同値であることを利用する.
\end{proof}
\section{圏論}
\subsection{極限}
\begin{definition}
    $C,D$を圏とする. 対角関手$\Delta: D \rightarrow D^{C}$とは
    \begin{itemize}
        \item $a \in D$に対して$\Delta a$は以下で与えられる関手$\Delta a: C \rightarrow D$である.
        \begin{itemize}
            \item $c \in C$に対して$\Delta a(c)=a$
            \item $f \in \textrm{Mor}(C)$に対して$\Delta a(f)=\textrm{id}_a$
        \end{itemize}
        \item $f : a \rightarrow b$に対して, $\Delta f$は「$c \in C$に対して, $(\Delta f)_c=f$」で与えられる
        自然変換$\Delta f: \Delta a \rightarrow \Delta b$である.
    \end{itemize}
\end{definition}
\begin{definition}
    $C,D$を圏, $c \in C$を対象, $G: D \rightarrow C$を関手とする. 以下を満たす組$\langle d,f\rangle$
    を$c$から$G$への普遍射という.
    \begin{enumerate}[(1)]
        \item $d$は$D$の対象である.
        \item $f$は$C$の射$f: c \rightarrow Gd$である.
        \item 組$\langle d', f' \rangle$が上2つの条件を満たすならば, $D$の射$h: d \rightarrow d'$が一意に存在して,
        $Gh \circ f=f'$となる.
    \end{enumerate}
\end{definition}
\begin{definition}
    $J, C$を圏として, $\Delta: C \rightarrow C^{J}$を対角関手とする.
    \begin{enumerate}[(1)]
        \item 関手$T:J \rightarrow C$を図式という. また$J$を図式$T$の添え字圏という.
        \item $\Delta$から$T \in C^{J}$への普遍射$\langle \textrm{lim}T, \pi \rangle$を図式$T$の極限という.
        \item $T \in C^{J}$から$\Delta$への普遍射$\langle \textrm{colim}T, \mu \rangle$を図式$T$の余極限という.
    \end{enumerate}
\end{definition}
\subsection{随伴写像}
\url{http://alg-d.com/math/kan_extension/adjoint.pdf}
\begin{definition}
    $C, D$を圏, $F: C \rightarrow D, G: D \rightarrow C$を関手とする. $c \in C, d \in D$について
    自然な全単射$\phi_{cd}: \mathrm{Hom}_D(Fc, d) \rightarrow \mathrm{Hom}_C(c,Gd)$が存在するとき,
    3つ組$\langle F,G,\phi \rangle$のことを随伴という. このとき記号では$F \dashv G: C \rightarrow D$
    もしくは単に$F \dashv G$と書く. また$F$を$G$の左随伴写像, $G$を$F$の右随伴写像という.
\end{definition}
\begin{definition}
    自然同型$\phi$により次のような二つの射が一対一に対応することになる.
    \[ f:Fc \rightarrow d, \ g:c \rightarrow Gd \]
    $\phi_{cd}(f) = g$のとき, $g$を$f$の右随伴射, $f$を$g$の左随伴射と呼ぶ.
    本PDFでは随伴射を$\tilde{\ }$と表すことにする. つまり$f:Fc \rightarrow d, g:
    c \rightarrow Gd$のとき, $\tilde{f}: c \rightarrow Gd, \tilde{g}: Fc \rightarrow d$
    であり, $\phi_{cd}(f) = \tilde{f}, \phi_{cd}(\tilde{g}) = g$である.
\end{definition}
\begin{thm}
    $f:Fc \rightarrow d, h:Fc' \rightarrow d', p: c \rightarrow c', q: d \rightarrow d'$
    とする. この時次の左の図式が可換ならば右の図式も可換であり, 右の図式が可換ならば, 左の図も
    可換である.
\begin{align*}
\xymatrix{
Fc \ar[r]^-{f} \ar[d]^-{Fp} & d \ar[d]^-{q}& 
Fc'\ar[r]^-{\tilde{f}} \ar[d]^{p} & d' \ar[d]^-{Gq} \\
c \ar[r]^-{h}& Gd
& c'\ar[r]^-{\tilde{h}}& Gd'
}
\end{align*}
\end{thm}
\begin{proof}
    $\phi_{cd}$の自然変換の可換図から. $c,d$それぞれ固定したときを考える.
    $q, Gq$の向きを入れ替えた図式についても成立する.
\end{proof}
\begin{definition}
    $d=Fc$とすると$\textrm{Hom}_D(Fc, Fc) \cong \textrm{Hom}_C(c, GFc)$
    となる.左辺の$\textrm{id}_{Fc}$について, $\eta_{c}:= \widetilde{\textrm{id}_{Fc}}$と定義する.
\end{definition}
\begin{thm}
    $\langle Fc, \eta_{c} \rangle$は$c$から$G$への普遍射である.
\end{thm}

\begin{cor}
    全単射$\textrm{Hom}_D(Fc, d) \rightarrow \textrm{Hom}_C(c, Gd)$
    は$f \mapsto Gf \circ \eta_c$で与えられる.
\end{cor}
\begin{thm}
    $G: D \rightarrow C$を関手として, 各$c \in C$に対して普遍射$\eta_c:c \rightarrow Gd_c$
    が存在するとする. このとき対応$c \mapsto d_c$は関手$F: C \rightarrow D$を定め, $F \dashv G$となる.
\end{thm}



\end{document}